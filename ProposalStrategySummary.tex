\documentclass[12pt]{article}

\usepackage[top=0.75in, bottom=0.75in, left=1in, right=1in]{geometry}
\pagestyle{empty}
\usepackage{tabu}

\begin{document}

\begin{center}
{Proposal Strategies, NSF IWBW Summary}
\end{center}

\setlength{\unitlength}{1in}
\begin{picture}(6,.1) 
\put(0,0) {\line(1,0){6.25}}         
\end{picture}

\renewcommand{\arraystretch}{2}
%\vspace{2em}
%\noindent \underline{Workshop goal}: enhance the participants' knowledge of essential elements of an educational proposal, or education component of more general proposal, and their understanding of strategies for developing more effective proposals. 

\vspace{2em}
\noindent Competitive proposals have
\begin{itemize}
\item an understanding of the review process reflected in them
\item great idea(s) that addresses important problems in the field, are embedded in relevant theory and literature, and are significant to relevant stakeholders
\item well designed project, including management and implementation plans
\item a coherent narrative that integrates each component of the project, including evaluation
\end{itemize}

\vspace{1em}
\noindent Review criteria -- must address \textbf{both}
\begin{enumerate}
\item intellectual merit (potential to advance knowledge)
\item broader impacts (potential to benefit society and contribute to the achievement of specific, desired outcomes)
\end{enumerate}

\vspace{1em}
\noindent General:
\begin{itemize}
\item What is the potential for the activity to meet these criteria?
\item Does it explore creative, original, or potentially transformative concepts?
\item Is it well-reasoned and well-organized? Can the project assess success?
\item How well qualified is the team?
\item Are their adequate resources?
\end{itemize}

\vspace{1em}
\noindent \underline{Elements of a competitive proposal:}
\vspace{1em}

\noindent * Outline goals and objectives (what)

\vspace{1em}
%%%%%%%
\noindent * Use rationale to provide (why)
\begin{itemize}
\item background and context
  \begin{itemize}
  \item how does the proposed work fit into and relate to prior
        work of others? Of the applicant?
  \item How does the proposed work fit into and relate to relevant
        theories?
  \end{itemize}
\item justification and significance
  \begin{itemize}
  \item How does the proposal incorporate the current understanding
        of teaching and learning?
  \item '' new disciplinary knowledge?
  \item '' address an emerging area or known problem?
  \item '' address an industry/scientific/societal need?
  \item How easily is the proposed work transportable to other
        institutions and contexts?
  \item What are the potential contributions to the teaching and
        learning knowledge base?
  \item What are the potential limitations of alternate approaches?
  \end{itemize}
\item connect the goals and objectives to the project plan
\item http://ies.ed.gov/pdf/CommonGuidelines.pdf (outlines types of research and associated expected justification and evidence)
\end{itemize}
%%%%%%%
* Project Plans (how)
\begin{itemize}
\item implementation plan (how and when things are done): strategies; products; equipment and resources; timeline\\
\emph{Timeline} shows connections between the other pieces of the plan (management, eval, etc.)  
\item management plan (who is doing something when): demonstrates responsibility and appropriateness $\rightarrow$ high probability of success.
 \begin{itemize}
 \item highlight qualifications of researchers
 \item each PI/co-PI has specific tasks
 \item \textbf{independent, external evaluator is used}
 \end{itemize}
\item dissemination plan
\item evaluation plan
\end{itemize}

\vspace{2em}
\noindent ----- Aside from the CommonGuidelines document -----\\
It looks like we're doing

\underline{Research Type \#5}: Effectiveness Research examines effectiveness of a strategy or intervention under circumstances that would typically prevail in the target context. The importance of ``typical" circumstances means that there should not be more substantial 
developer support than in normal implementation, and there should not be substantial developer involvement in the evaluation of the strategy or intervention.

\underline{Impact Research} includes designs that eliminate or reduce bias arising from self-selection into treatment and control 
conditions, clearly specified outcome measures, adequate statistical power to detect effects, and data on implementation of the intervention or strategy and the counterfactual condition.

\end{document}