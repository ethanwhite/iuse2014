\documentclass{proposalnsf}

% Document:	NSF proposal
% Due:		Feb. 17, 2012
% Author:		Lorena Barba
% http://www.nsf.gov/funding/pgm_summ.jsp?pims_id=13365&org=CBET&from=home

%--------------------------------------------------------------------  PROCESS WITH XeLaTeX
\usepackage{fontspec}% provides font selecting commands 
\usepackage{xunicode}% provides unicode character macros 
\usepackage{xltxtra} % provides some fixes/extras 
\setromanfont[Mapping=tex-text,
                 SmallCapsFont={Palatino},
                 SmallCapsFeatures={Scale=0.85}]{Palatino}
\setsansfont[Scale=0.85]{Trebuchet MS} 
\setmonofont[Scale=0.85]{Monaco}

\renewcommand{\captionlabelfont}{\bf\sffamily}
\usepackage[hang,flushmargin]{footmisc} 
% 'hang' flushes the footnote marker to the left,  'flushmargin'  flushes the text as well.



% Define the color to use in links:
\definecolor{linkcol}{rgb}{0.459,0.071,0.294}
\definecolor{sectcol}{rgb}{0.63,0.16,0.16} % {0,0,0}
\definecolor{propcol}{rgb}{0.75,0.0,0.04}

\definecolor{blue}{rgb}{0,0,0}
\definecolor{green}{rgb}{0.5,0.5,0.5}
\definecolor{gray}{rgb}{0.25,0.25,0.25}
\definecolor{ngreen}{rgb}{0.7,0.7,0.7} % a darker shade of green



\usepackage[
    xetex,
    pdftitle={NSF proposal},
    pdfauthor={Lorena Barba},
    pdfpagemode={UseOutlines},
    pdfpagelayout={TwoColumnRight},
    bookmarks, bookmarksopen,bookmarksnumbered={True},
    pdfstartview={FitH},
    colorlinks, linkcolor={sectcol},citecolor={sectcol},urlcolor={sectcol}
    ]{hyperref}

%% Define a new style for the url package that will use a smaller font.
\makeatletter
\def\url@leostyle{%
  \@ifundefined{selectfont}{\def\UrlFont{\sf}}{\def\UrlFont{\small\ttfamily}}}
\makeatother
%% Now actually use the newly defined style.
\urlstyle{leo}


% this handles hanging indents for publications
\def\rrr#1\\{\par
\medskip\hbox{\vbox{\parindent=2em\hsize=6.12in
\hangindent=4em\hangafter=1#1}}}


\addto\captionsamerican{%
  \renewcommand{\refname}%
    {References Cited}%
} % solution found here: http://www.tex.ac.uk/cgi-bin/texfaq2html?label=latexwords

\newcommand{\nvidia}{\textsc{NVIDIA}\xspace}
\newcommand{\petsc}{\textsc{PETS}c\xspace}
\newcommand{\mpi}{\textsc{MPI}\xspace}
\newcommand{\cfd}{\textsc{CFD}\xspace}
\newcommand{\simd}{\textsc{SIMD}\xspace}
\newcommand{\openmp}{\textsc{O}pen\textsc{MP}\xspace}
\newcommand{\ibm}{\textsc{IBM}}
\newcommand{\pde}{\textsc{PDE}}
\newcommand{\cpu}{\textsc{CPU}}
\newcommand{\gpu}{\textsc{GPU}}
\newcommand{\cuda}{\textsc{CUDA}\xspace}
\newcommand{\mg}{\textsc{MG}\xspace}
\newcommand{\cfl}{\textsc{CFL}\xspace}

\def\baselinestretch{1}
\setlength{\parindent}{0mm} \setlength{\parskip}{0.8em}

\newlength{\up}
\setlength{\up}{-4mm}

\newlength{\hup}
\setlength{\hup}{-2mm}

\sectionfont{\large\bfseries\color{sectcol}\vspace{-2mm}}
\subsectionfont{\normalsize\it\bfseries\vspace{-4mm}}
\subsubsectionfont{\normalsize\mdseries\itshape\vspace{-4mm}} %\itshape
\paragraphfont{\bfseries}
% ---------------------------------------------------------------------
\begin{document}


% ------------------------------------------------------------------- Biosketch
%\newpage
\pagenumbering{arabic}
\renewcommand{\thepage} {\footnotesize Bio.\,---\,\arabic{page}}
\section*{Prof.\ Lorena Barba}

\small
\textbf{Education and training:} 

\begin{tabular}{llcc}
Institution & Major & Degree & Year \\ \hline
Universidad Tecnica Federico Santa Maria, Chile   &   Mechanical Engineering & BSc & 1989 \\
Universidad Tecnica Federico Santa Maria, Chile   &   Mechanical Engineering & PEng & 1998 \\
California Institute of Technology, Pasadena, CA   &   Aeronautics & MSc & 1999 \\
California Institute of Technology, Pasadena, CA  &   Aeronautics & PhD & 2004 \\
\end{tabular}

\textbf{Research and professional experience:} 

09/2008-- \qquad Assistant Professor, Mechanical Engineering, Boston University

08/2004--09/2008 \qquad Lecturer in Applied Mathematics, University of Bristol, UK


\textbf{Honors and awards:} 

02/2012 \quad  National Science Foundation CAREER award.

08/2011 \quad  NVIDIA Academic Partnership award (\textdollaroldstyle 25,000 unrestricted cash award)
  
01/2008 \quad  \textit{Rising Star} Teaching Award for the Faculty of Science,
  University of Bristol
  
3/2005 \quad   The Nuffield   Foundation Award to Newly Appointed Lecturers in
  Science, Engineering and Mathematics (\pounds 5000)

1999 \qquad  \textit{Amelia Earhart Fellowship Award}, for aerospace-related
  graduate studies at doctoral level, Zonta International Foundation (\textdollaroldstyle 6000).





\textbf{Publications:} % -------------------------------------------- 
%
\vspace{\up}
\begin{list}{$\ast$}{\setlength{\leftmargin}{1em}}

\item  Simon K. Layton, Anush Krishnan,  L.~A. Barba, cuIBM---A GPU-accelerated immersed boundary method, Submitted 2011. Preprint on \href{http://arxiv.org/abs/1109.3524}{arXiv:1109.3524}

\item  Rio Yokota, L.~A.~Barba, Tetsu Narumi, Kenji Yasuoka, Petascale turbulence simulation using a highly parallel fast multipole method, \textit{Computer Physics Communications}.  Accepted, under revision. Preprint on \href{http://arxiv.org/abs/1106.5273}{arXiv:1106.5273}

\item  Rio Yokota, L.~A.~Barba, A tuned and scalable fast multipole method as a preeminent algorithm for exascale systems, \textit{Int.\ J.\ High-Performance Computing and Applications}. In press; published online Jan.\ 2012, \href{http://hpc.sagepub.com/content/early/2012/01/18/1094342011429952.abstract}{doi:10.1177/1094342011429952}

\item Felipe A.~Cruz, Simon K.~Layton, L.~A.~Barba, How to obtain efficient \gpu\ kernels: an illustration using FMM \& FGT, \textit{Computer Physics Communications}, \textbf{182}(10):2084--2098 (2011) \href{http://dx.doi.org/10.1016/j.cpc.2011.05.002}{doi:10.1016/j.cpc.2011.05.002}

\item Rio Yokota, Jaydeep P.~Bardhan, Matthew G.~Knepley, L.~A.~Barba, Tsuyoshi Hamada, Biomolecular electrostatics using a fast multipole BEM on up to 512 \gpu s and a billion unknowns, \textit{Computer Physics Communications}, \textbf{182}(6):1271--1283 (2011) \href{http://dx.doi.org/10.1016/j.cpc.2011.02.013}{doi:10.1016/j.cpc.2011.02.013}

\item Rio Yokota, L.~A.~Barba, Treecode and fast multipole method for $N$-body simulation with CUDA, Ch.~9 in \textit{\gpu\ Gems Emerald Edition}, Wen-mei Hwu, ed.; Morgan Kaufmann/Elsevier (Jan.~2011), pp.~113--132. Preprint on \href{http://arxiv.org/abs/1010.1482}{arXiv:1010.1482}

\item  Felipe A.~Cruz, L.~A.~Barba, Matthew G.~Knepley, Pet\textsc{FMM}---a dynamically load-balancing parallel fast multipole library.   \textit{Int.\ J.\ Num.\ Meth.\ Fluids}; \href{http://dx.doi.org/10.1002/nme.2972}{doi:10.1002/nme.2972} (2010).

\item Rio Yokota, L.~A.~Barba, Matthew G.~Knepley, Pet\textsc{RBF}---a parallel $\mathcal{O}(N)$ algorithm for radial basis function interpolation, \textit{Comput.\ Meth.\ Appl.\ Mech.\ Eng.}, \textbf{199}(25--28):1793--1804 \\ 
\href{http://dx.doi.org/10.1016/j.cma.2010.02.008}{doi:10.1016/j.cma.2010.02.008} (2010).

%\item L.~A.~Barba, A.~Leonard, C.~B.~Allen, Advances in viscous vortex methods---Meshless spatial adaption based on radial basis function interpolation, \textit{Int.\ J.\ Num.\ Meth.\ Fluids} \textbf{47}(5): 387--421, \href{http://dx.doi.org/10.1002/fld.811}{doi: 10.1002/fld.811} (2005).

\end{list}

\newpage

\textbf{Synergistic activities:} % --------------------------------------------  

\vspace{\up}


\begin{list}{ }{\setlength{\leftmargin}{2.5em}}


	\item[]  Dr Barba has consistently advocated and participated in the open science philosophy.  Within her research group, all software developed is open-source and publicly available, at the different stages of development.  Moderate-size codes are uploaded to \href{http://code.google.com/u/lorena.barba/}{Google code} at the time of submission of a manuscript, while the preprint is uploaded to the \href{http://arxiv.org/find/cs/1/au:+Barba_L/0/1/0/all/0/1}{Arxiv} repository.  More complex software projects have a public version-controlled repository, while documentation and links are provided on an informative and well-maintained \href{http://barbagroup.bu.edu/}{group website}.

	Consistent with the open science commitment, Dr Barba also has engaged in the open course-ware movement.  Her latest two courses taught at Boston University have been made available publicly in the form of (video) screencasts via \href{http://deimos3.apple.com/WebObjects/Core.woa/Browse/bu.edu.3295784150}{iTunes U}.
	
	\item[Jan.'11] \emph{``PASI---Pan-American Advanced Studies Institute.  Scientific Computing in the Americas: The challenge of massive parallelism''} (Valpara\'iso, Chile).  Two-week advanced studies institute, organized entirely by Dr Barba and funded by NSF via award OISE-1036435 (amount \$100,600).  The focus of this PASI is on scientific discovery by means of high-performance computing, HPC,  using the latest many-core computer hardware, in particular graphics processors, or \gpu s.  Dr Barba has been promoting \gpu\ technology with her contacts in Latin America, as a technology that can level the playing field for countries of the ``scientifically developing'' category.  Bringing this event to Chile aims at building scientific capacity, promoting collaboration, and advanced training of young researchers in HPC. 
	
	\item[Nov.'05--Jul.'09] \emph{``SCAT---Scientific Computing Advanced Training''}: Dr Barba put together an international network of collaboration involving 10 institutions in 6 countries, and prepared a proposal to the European Commission co-operation office for a grant under \emph{Programme ALFA II}, for projects of collaboration of higher-education institutions in Europe and Latin America, for scientific and technical training and knowledge transfer.   The proposal was  successful and the project was awarded a budget of nearly \textbf{\euro 1.4 million} for the whole duration.  It was one of only six projects funded in 2005 within this program.
The SCAT project involved more than 40 professors and researchers, it awarded nearly 30 grants for graduate students and postdocs to travel to a guest institution within the network to carry out research as a visiting scholar (for a collective 235 funded graduate-student months), and sponsored 10 international scientific meetings. Dr Barba hosted 6 of the grantees at Bristol, 4 of which continued studies in PhD programs around the world. More information on the website:  \url{http://www.scat-alfa.eu}


	\item[\emph{Invited Seminars:}] Dr Barba has visited the following institutions to give seminars: \vspace{\hup}
\item[In 2010:] Columbia University, Brown University, Worcester Polytechnic Institute.\vspace{\hup}
\item[2008--'09:] Harvard University (Initiative in Innovative Computing, IIC), Illinois Institute of Technology, Purdue University, Northwestern University, University of Sussex UK, Instituto Madrileno de Estudios Avanzados IMDEA, Madrid.\vspace{\hup}
\item[2004--'07] University of Delaware, Exeter University UK, Technical University of Eindhoven, Johns Hopkins University,  Bath University UK, ETH Zurich (Institute of Computational Science), University of Southampton UK, University of Leicester UK (Centre for Mathematical Modelling). 
\end{list}




%%%%%%%%
\end{document}
