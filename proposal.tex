\documentclass[11pt]{article}

\title{
  Assessing the Impact of Intensive Software Skills Training
  on Students' Scientific Careers
}

\author{
  Rachel Slaybaugh, Lorena Barba, C. Titus Brown, Ethan White, and Paul Wilson\\
  with\\
  Scott Collins, Kaitlin Thaney, Tracy Teal, and Greg Wilson
}

\newcommand{\fixme}[2]{FIXME (#1): {#2}}

\begin{document}
\maketitle
\pagebreak

\section{Summary}

FIXME: one-page summary

\pagebreak

\section{The Problem}

Scientists and engineers invented electronic computers to accelerate
their work, but two generations later, many researchers in science,
technology, engineering, and mathematics (STEM) are still not
\emph{computationally competent}: they do repetitive tasks manually
instead of automating them, develop software using a methodology best
summarized as ``copy, paste, tweak, and pray'', and fail to track
their work in any systematic, reproducible way.

Equally, while the World-Wide Web was created by a scientist to help
his peers share information, many still use it primarily as a way to
find and download PDFs.  Researchers may understand that open data can
fuel new insights, but often lack the skills needed to create and
provide a reusable data set.  Equally, any discussion of changing
scientific publishing, make research reproducible, or using the web to
support ``science as a service'' must eventually address the lack of
pre-requisite skills in the general STEM research community.

Since the mid-1980s (at least), proponents of computational science
have taken an ``if we build it, they will come'' approach to this
problem.  It is clear now, though, that learning-by-osmosis has not
worked, and is unlikely to in future for several reasons:

\begin{enumerate}

\item
  \emph{The curriculum is full.}  Undergraduate STEM programs already
  struggle to cover material regarded as core to their field.  While
  many scientists would agree that more material on programming,
  reproducible research, or web-enabled science would be useful, there
  is no consensus on what to take out to make room.

\item
  \emph{The blind leading the blind.}  Many faculty lack these skills
  themselves, and hence are unable to pass them on.

\item
  \emph{Difficulty of assessing impact.} It is easy to say, ``This
  discovery could not have been made without use of that
  supercomputer.''  It is much harder to attribute specific advances
  in science to prior training in general computing skills.

\end{enumerate}

The final, and possibly largest, issue is that \emph{the rewards are
  unknown}.  Open, web-based science is still in its infancy, so there
is no general understanding of what people might need to know in order
to incorporate it into their research careers.  Since it is hard to
measure something if you don't know what to look for, or if it is so
young that there hasn't actually \emph{been} long-term impact, no
systematic study has been to date of whether early training in the
skills needed to practice open, web-enabled science actually has an
impact, and if so, how and how much.  Without such feedback, there is
no systematic way to improve the training programs that currently
exist.

\section{Related Work}

\subsection{Research}

Studies of how scientists use computers and the web have found that
most scientists learn what they know about developing software and
using computers and the web in their research through osmosis and word
of mouth. Most training meant to address this issue:

\begin{itemize}

\item
  does not target scientists' specific needs (e.g., is a general
  ``Introduction to Computing'' class shared with students majoring in
  other areas);

\item
  only covers the mechanics of programming in a particular language
  rather than giving a complete picture including data management,
  testing, publishing, and other tasks; and/or

\item
  jumps to advanced topics such as parallel computing before
  scientists have mastered the foundations.

\end{itemize}

\subsection{Software Carpentry}

Software Carpentry is the largest effort to date to address these
issues. Originally created as a training program at Los Alamos
National Laboratory in the late 1990s, it is now part of the Mozilla
Science Lab's efforts to help scientists take advantage of ways in
which the web can change the practice of science today, and invent new
ways tomorrow.  Over 100 certified volunteer instructors delivered
two-day intensive workshops to more than 4200 people in 2013 alone.
Their ostensible subject matter usually includes:

\begin{itemize}
\item
  the Unix shell,
\item
  version control with Git,
\item
  programming in Python or R, and
\item
  using SQL.
\end{itemize}

What these workshops actually seek to convey, though, is:

\begin{itemize}
\item
  how to automate repetitive tasks,
\item
  how to track and share work,
\item
  how to grow a program in a modular, testable, reusable way, and
\item
  the difference is between structured and unstructured data.
\end{itemize}

Software Carpentry's curriculum and teaching practices have been
refined via iterative design, and are informed by current research on
teaching and learning best practices.  Its instructor training program
introduces participants to a variety of modern teaching techniques
(e.g., peer instruction), to concepts underlying these techniques
(e.g., cognitive load theory), and to topic-specific work by computing
education researchers including Guzdial \& Ericson (Georgia Tech),
Simon (UC San Diego), and Sorva (Aalto University).  One example of
how we translate theory into practice is our insistence on live coding
during teaching as a way of demonstrating and transferring authentic
practice to learners.

Software Carpentry has been assessing learning outcomes and retention
since the beginning of its Sloan Foundation-funded in January 2012.
The first round of assessment included both qualitative and
quantitative assessment by Dr.\ Jorge Aranda (then at the University
of Victoria) and Prof.\ Julie Libarkin (Michigan State University).
An attempt to scale this up in 2013 was set back by personnel changes,
but systematic follow-ups with past participants in workshops have now
been resumed, and we expect to be able to present preliminary results
by mid-2014.

\subsection{The Hacker Within}

\fixme{PW/KH}{Please fill in.}

\section{Proposed Work}

This proposal builds on our success to date in enhancing the skills of
graduate students, post-docs, and faculty.  We designed it to:

\begin{enumerate}

\item
  conduct formative evaluation of the impact of software skills
  training for undergraduates likely to continue in research careers
  as they progress through the early stages of those careers;

\item
  conduct summative evaluation of the training's overall impact on a
  multi-year timescale in order to improve the content and
  presentation of the training; and

\item
  disseminate the resulting curriculum as widely as possible.

\end{enumerate}

More specifically, we plan to run two-day software skills workshops
for undergraduate students taking part in the NSF's Research
Experience for Undergraduates (REU) program each year for four years,
at or near the start of their time in the lab.  We believe this
training will help them be more productive during their REU
(graduate-level participants in our existing workshops typically
report that what we teach saves them a day a week), and also prepare
them to work in a world where all aspects of science are increasingly
dependent on computing.  More importantly, these undergraduates will
serve as the treatment population for a five-year study of the impact
of this training on their careers in general, and their involvement
with open and web-enabled science in particular.

The sections below detail the specific activities we will undertake.

\subsection{Workshops}

We will run two-day workshops at a steadily increasing number of sites
each year for four years, timed to coincide with the start of the
summer REU influx.  Each workshop will be offered to a minimum of 40
learners per site (giving is a target study population of 1440
students by year 4).  The content will be tailored to meet local
needs, but will be based on what is being used at that time by
Software Carpentry and affiliated educational efforts.  All of the
workshop instructors will have been trained and certified by Software
Carpentry, and will have had prior experience teaching this material.

The six home sites for investigators named in this proposal (Michigan
State U., Utah State U., George Washington U., U.\ New Mexico, UC
Berkeley, and U.\ Wisconsin - Madison) will run workshops in each of
those years.  Two other NSF REU sites will be added each year,
increasing the total to 12 sites by year 4, to increase the size of
our study population.

In order to increase the diversity of the study population, we will
additionally run at least one workshop in each of years 1-4 that is
specifically aimed at female students.  Software Carpentry's first
such workshop, held in Boston in June 2013, attracted 120
participants; its second is scheduled for Lawrence Berkeley National
Laboratory in March 2014, and at least two more will be held by the
time work on this project commences (one in the US, and one in
Europe).  This work will build on that experience, and draw on the
pool of instructors who have gained mentoring experience through those
specific workshops.

Finally, we will organize an equal number of workshops specifically
aimed at students from minority groups that are under-represented in
STEM.  We are already in contact with the Computing Alliance for
Hispanic-Serving Institutions (CAHSI), and with the Association of
Public and Land-grant Universities' program for historically black
colleges and universities (HBCUs); Software Carpentry is running its
first workshop at an HBCU (Spelman) in early 2014, and we expect to
have expanded these efforts by the start of this project.

\subsection{Assessment}

We will expand our assessment efforts to compare REU student outcomes
with those of other learners, and to see what impact this training has
on them compared with non-participant peers.  More specifically, we
will employ one full-time researcher for five years to monitor
undergraduate participants in these workshops, participants in a
subset of our regular (graduate-level) workshops, and non-participants
(as a control population) for comparison purposes.  This researcher
will also explore ways in which our engagement with students changes
the outlook and work practices of their peers and faculty supervisors,
i.e., whether there is knowledge transfer sideways and upward.

As with our work to date, assessment will use both qualitative and
quantitative techniques.  On the qualitative side, we will conduct a
series of interviews over the five-year period of the study to see how
attitudes, aspirations, and activities change.  Quantitatively, we
will measure uptake of key tools such as version control as a proxy
for adoption of related practices, as well as exploring more
traditional measures of research success such as progression to
graduate school and publication/citation rates.

\subsection{Community Building}

We will employ one graduate student part-time at each participating
site each year to provide technical support to workshop participants,
and to act as an anchor for a Hacker Within-style grassroots group at
that site.  These community liaisons will not be study subjects, but
will help us stay in touch with students who are (a key requirement
for any longitudinal study).

Separately, the Mozilla Science Lab will focus part of its ongoing
community engagement efforts on the students who have taken part in
our workshops during both the remainder of their undergraduate careers
and afterward in order to ensure that they become part of the broader
open science community.  This may include helping the students
organize and run workshop of their own in subsequent years, connecting
them with other open science projects, introducing them to potential
graduate supervisors who understand and value their new skills and
outlook, etc.

\subsection{Curriculum Development and Dissemination}

We will employ one instructional designer part-time during each of the
study's first four years to create new material, and to improve
existing material based on feedback from workshop participants and the
assessment program.  Here, ``creating material'' may include both
designing and implementing new domain-specific learning modules, and
translating existing materials into new forms, such as video
recordings of lectures or auto-graded quizzes for self-paced
instruction.  This work will be done in consultation with educators at
participating institutions in order to encourage incorporation of
those materials into existing curricula.

All of the materials produced by and for this project will be made
freely available under the Creative Commons - Attribution (CC-BY)
license.  The instructional designer will work with the Mozilla
Science Lab and affiliated groups to share these materials, and the
results of our studies of the program's impact, through science
education journals, conferences, and other channels.

\section{Broader Impact}

We believe this work will have significant impact in several related
areas.

\begin{enumerate}

\item
  \emph{Improving STEM education for everyone, not just participants.}
  By creating and validating high-quality open access teaching
  materials, and the methods used to deliver them, this project will
  enable improvement in STEM education for everyone, everywhere, not
  just for participating students and participating institutions.

\item
  \emph{Improving STEM education tomorrow, not just today.}  As noted
  in the introduction of this proposal, most of today's efforts to
  transfer computational skills to STEM researchers and connect them
  with 21st Century innovations in how science is done are flying
  blind: there is effectively no feedback from long-term impact to
  instructional action.  By creating and validating such a feedback
  loop---i.e., by showing scientists how to apply science to their
  teaching---this project will demonstrate how STEM education can be
  continuously improved.

\item
  \emph{Enhance economic competitiveness.} Computing is no longer
  optional in any part of science: even scientists who don't think of
  themselves as doing computational work rely on computers to prepare
  papers, store data, and collaborate with colleagues.  The better
  their computing skills are, the better able they will be to
  contribute to the research that underpins the nation's economic
  competitiveness.

\item
  \emph{Improve participation in STEM by women and under-represented
    minorities.} The disproportionately low participation of women and
  some minority groups in STEM is well documented, as is the fact that
  computing is one of the least diverse fields within STEM.  This
  second fact creates a vicious circle: people with weaker computing
  skills may be less competitive in research than their peers, which
  reduces their participation in activities viewed as non-core, which
  in turn results in them having weaker skills.  This project will
  strive to break this cycle by giving at-risk students an opportunity
  to ``level up'' in a supportive environment, and by connecting them
  with mentors who can serve as role models.

\end{enumerate}

\section{Career Management Plan}

\fixme{EW}{Talk about the part-time grad students we're planning to hire.}

\section{Data Management Plan}

\fixme{TT}{Talk about curating our data.}

\end{document}
